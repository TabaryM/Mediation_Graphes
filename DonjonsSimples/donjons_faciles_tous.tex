\documentclass{article}

\usepackage[french]{babel} % pour qu'il sache qu'on écrit du français
                           % (et s'adapte sur les espaces etc.)
\usepackage[utf8]{inputenc} %encodage
\usepackage{a4wide} % pour avoir de la place sur la feuille (de
                    % petites marges)
\usepackage{amsmath}% je l'ai utilisé pour la commande boldsymbol,
                    % contient plein de symboles matheux
\usepackage{pgf} % celui-ci et le suivant sont pour les graphes
\usepackage{tikz}

\usepackage{fancyhdr} % Pour la mise en page des en-tête et pieds de pages
% Package ajoutant la licence de partage et d'utilisation du document
\usepackage[
    type={CC},
    modifier={by-sa},
    version={4.0},
]{doclicense}

% Pour ajouter le logo aux pieds de pages
\usepackage[scale=2.0]{ccicons}

% Creation du pied de page :
\pagestyle{fancy}
\lhead{}
\chead{}
\rhead{}
\lfoot{Mathieu Tabary}
\cfoot{\thepage}
\rfoot{\ccbysa}
\renewcommand{\headrulewidth}{0pt}
\renewcommand{\footrulewidth}{0pt}

\usetikzlibrary{positioning}% pour placer un noeud par rapport a un autre

% les deux lignes ci-dessous définissent une macro : après ça je peux
% utiliser \flg et \fld pour des flèches vers la gauche ou la droite
\newcommand{\flg}{$\boldsymbol{\leftarrow}$}
\newcommand{\fld}{$\boldsymbol{\rightarrow}$}

\begin{document}
\section{Enoncés}
    % Donjon 1
    \subsection{Donjon 1}
        Placez l'entrée en haut à gauche de la feuille.
        \bigskip

        %Rose des vents
        \begin{tikzpicture}[rotate=90, scale=0.35, every node/.style={scale=0.6}]
            \foreach   \direction/\label in {0/N, 90/E, 180/S, 270/W}{
                \node[fill=green!10] at ( -\direction : 15mm) (\label) {\label};
            }
            \draw[arrows={->[scale=1]}] (N) -- (S);
            \draw[arrows={->[scale=1]}] (S) -- (N);
            \draw[arrows={->[scale=1]}] (E) -- (W);
            \draw[arrows={->[scale=1]}] (W) -- (E);
        \end{tikzpicture}
        \bigskip

        \begin{tikzpicture}
              \tikzstyle{entree}=[circle, thick, draw=blue!75, fill=blue!10, minimum width=2.5cm, align=center]
              \tikzstyle{salle}=[circle, draw, minimum width=1.5cm, align=center]

              \node[entree] (E0) {Entrée};
              \node[salle, right=of E0] (M) {M};
              \node[salle, right=of M] (B) {B};
              \node[salle, right=of B] (J) {J};
              \node[salle, below=of M] (D) {D \\ Trésor};
              \node[salle, below=of J] (U) {U};

              \draw (E0) to (M);
              \draw (M) to (D);
              \draw (M) to (B);
              \draw (B) to (J);
              \draw (J) to (U);
              \draw [bend right=45](B) to (U);
        \end{tikzpicture}

    \newpage
    % Donjon 2
    \subsection{Donjon 2}
        Placez l'entrée au milieu à gauche de la feuille.
        \bigskip

        %Rose des vents
        \begin{tikzpicture}[rotate=90, scale=0.35, every node/.style={scale=0.6}]
            \foreach   \direction/\label in {0/N, 90/E, 180/S, 270/W}{
                \node[fill=green!10] at ( -\direction : 15mm) (\label) {\label};
            }
            \draw[arrows={->[scale=1]}] (N) -- (S);
            \draw[arrows={->[scale=1]}] (S) -- (N);
            \draw[arrows={->[scale=1]}] (E) -- (W);
            \draw[arrows={->[scale=1]}] (W) -- (E);
        \end{tikzpicture}
        \bigskip


        \begin{tikzpicture}
              \tikzstyle{entree}=[circle, thick, draw=blue!75, fill=blue!10, minimum width=2.5cm, align=center]
              \tikzstyle{salle}=[circle, draw, minimum width=1.5cm, align=center]
              \node[entree] (E0) {Entrée};
              \node[salle, right=of E0] (C) {C};
              \node[salle, above=of C] (J) {J};
              \node[salle, below right=of C] (X) {X};
              \node[salle, above right=of X] (S) {S \\ Trésor};
              \node[salle, above=of S] (B) {B};
              \node[salle, right=of S] (H) {H};
              \node[salle, below=of X] (G) {G \\ Trésor};

              \draw (E0.north) .. controls +(up:1.25cm) .. (J.west);
              \draw (J) to (B);
              \draw (E0) to (C);
              \draw (C) to (S);
              \draw (S) to (H);
              \draw (B.east) .. controls +(right:1.75cm) .. (H.north);
              \draw (X) to (G);
              \draw (E0.south) .. controls +(down:0.75cm) .. (X.west);
              \draw (H.south) .. controls +(down:1.25cm) .. (X.east);
        \end{tikzpicture}

    \newpage
    % Donjon 3
    \subsection{Donjon 3}
        Placez l'entrée au milieu à gauche de la feuille.
        \bigskip

        %Rose des vents
        \begin{tikzpicture}[rotate=90, scale=0.35, every node/.style={scale=0.6}]
            \foreach   \direction/\label in {0/N, 90/E, 180/S, 270/W}{
                \node[fill=green!10] at ( -\direction : 15mm) (\label) {\label};
            }
            \draw[arrows={->[scale=1]}] (N) -- (S);
            \draw[arrows={->[scale=1]}] (S) -- (N);
            \draw[arrows={->[scale=1]}] (E) -- (W);
            \draw[arrows={->[scale=1]}] (W) -- (E);
        \end{tikzpicture}
        \bigskip


        \begin{tikzpicture}
              \tikzstyle{entree}=[circle, thick, draw=blue!75, fill=blue!10, minimum width=2.5cm, align=center]
              \tikzstyle{salle}=[circle, draw, minimum width=1.5cm, align=center]
              \node[entree] (E0) {Entrée};
              \node[salle, right=of E0] (A) {A};
              \node[salle, above=of A] (T) {T};
              \node[salle, right=of A] (F) {F};
              \node[salle, left=of T] (H) {H \\ Trésor};
              \node[salle, right=of T] (C) {C};
              \node[salle, right=of F] (G) {G \\ Trésor};
              \node[salle, right=of C] (N) {N};

              \draw (H) to (T);
              \draw (T) to (C);
              \draw (A) to (T);
              \draw (C) to (N);
              \draw (E0) to (A);
              \draw (A) to (F);
              \draw (F) to (G);
              \draw (N.south) .. controls +(down:0.75cm) and +(up:0.75cm) .. (F.north);
        \end{tikzpicture}

    \newpage
    % Donjon 4
    \subsection{Donjon 4}
        Placez l'entrée au milieu à gauche de la feuille.
        \bigskip

        %Rose des vents
        \begin{tikzpicture}[rotate=90, scale=0.35, every node/.style={scale=0.6}]
            \foreach   \direction/\label in {0/N, 90/E, 180/S, 270/W}{
                \node[fill=green!10] at ( -\direction : 15mm) (\label) {\label};
            }
            \draw[arrows={->[scale=1]}] (N) -- (S);
            \draw[arrows={->[scale=1]}] (S) -- (N);
            \draw[arrows={->[scale=1]}] (E) -- (W);
            \draw[arrows={->[scale=1]}] (W) -- (E);
        \end{tikzpicture}
        \bigskip


        \begin{tikzpicture}
            \tikzstyle{entree}=[circle, thick, draw=blue!75, fill=blue!10, minimum width=2.5cm, align=center]
            \tikzstyle{salle}=[circle, draw, minimum width=1.5cm, align=center]
            \node[entree] (E0) {Entrée};
            \node[salle, right=of E0] (N) {N};
            \node[salle, right=of N] (B) {B \\ Trésor};
            \node[salle, above=of B] (Y) {Y};
            \node[salle, below=of N] (D) {D};
            \node[salle, right=of D] (E) {E};
            \node[salle, below=of D] (J) {J};

            \draw (E0) to (N);
            \draw (N) to (B);
            \draw (N) [bend left=45] to (Y);
            \draw (B) to (Y);
            \draw (B.south) .. controls +(down:0.75cm) and +(up:0.75cm) .. (D.north);
            \draw (E0) [bend right=40] to (D);
            \draw (D) to (E);
            \draw (D) to (J);
            \draw (E) [bend left=45] to (J);
        \end{tikzpicture}

    \newpage
    % Donjon 5
    \subsection{Donjon 5}
        Placez l'entrée au milieu à gauche de la feuille.
        \bigskip

        %Rose des vents
        \begin{tikzpicture}[rotate=90, scale=0.35, every node/.style={scale=0.6}]
            \foreach   \direction/\label in {0/N, 90/E, 180/S, 270/W}{
                \node[fill=green!10] at ( -\direction : 15mm) (\label) {\label};
            }
            \draw[arrows={->[scale=1]}] (N) -- (S);
            \draw[arrows={->[scale=1]}] (S) -- (N);
            \draw[arrows={->[scale=1]}] (E) -- (W);
            \draw[arrows={->[scale=1]}] (W) -- (E);
        \end{tikzpicture}
        \bigskip


        \begin{tikzpicture}
              \tikzstyle{entree}=[circle, thick, draw=blue!75, fill=blue!10, minimum width=2.5cm, align=center]
              \tikzstyle{salle}=[circle, draw, minimum width=1.5cm, align=center]
              \node[entree] (E0) {Entrée};
              \node[salle, right=of E0] (G) {G};
              \node[salle, above=of G] (U) {U};
              \node[salle, left= 1.5cm of U] (A) {A};
              \node[salle, right=of U] (J) {J \\ Trésor};
              \node[salle, below=of G] (F) {F};
              \node[salle, left= 1.5cm of F] (R) {R};

              \draw (E0) to (G);
              \draw (G) to (U);
              \draw (G) to (F);
              \draw (U) to (A);
              \draw (U) to (J);
              \draw (E0) to (R);
              \draw (R) to (F);
              \draw (J.south) .. controls +(down:4.25cm) .. (F.east);
        \end{tikzpicture}

\newpage
\section{Corrections}
    % Donjon 1
    \subsection{Donjon 1}
        %Rose des vents
        \begin{tikzpicture}[rotate=90, scale=0.35, every node/.style={scale=0.6}]
            \foreach   \direction/\label in {0/N, 90/E, 180/S, 270/W}{
                \node[fill=green!10] at ( -\direction : 15mm) (\label) {\label};
            }
            \draw[arrows={->[scale=1]}] (N) -- (S);
            \draw[arrows={->[scale=1]}] (S) -- (N);
            \draw[arrows={->[scale=1]}] (E) -- (W);
            \draw[arrows={->[scale=1]}] (W) -- (E);
        \end{tikzpicture}

        \bigskip
        {\bf Parcours en largeur}

        \begin{tikzpicture}
            \tikzstyle{entree}=[circle, thick, draw=blue!75, fill=blue!10, minimum width=2.5cm, align=center]
            \tikzstyle{salle}=[circle, draw, minimum width=1.5cm, align=center]

            \node[entree, label=135:0 ,label=45:0] (E0) {Entrée};
            \node[salle, label=135:1 ,label=45:1 , right=of E0] (M) {M};
            \node[salle, label=135:2 ,label=45:2 , right=of M] (B) {B};
            \node[salle, label=135:3 ,label=45:4 , right=of B] (J) {J};
            \node[salle, label=135:2 ,label=45:3 , below=of M] (D) {D \\ Trésor};
            \node[salle, label=135:3 ,label=45:5 , below=of J] (U) {U};

            \draw (E0) to node [near start,sloped,above] {\fld} (M);
            \draw (M) to node [near start,sloped,above] {\fld} (D);
            \draw (M) to node [near start,sloped,above] {\fld} (B);
            \draw (B) to node [near start,sloped,above] {\fld} (J);
            \draw (J) to (U);
            \draw [bend right=45](B) to node [very near start,sloped,below] {\fld} (U);
        \end{tikzpicture}

        \bigskip
        {\bf Parcours en profondeur}\\
        Nombre de portes traversées: \\
        En suivant l'algorithme : 10 \\
        Au mieux : 7
        \bigskip

        \begin{tikzpicture}
            \tikzstyle{entree}=[circle, thick, draw=blue!75, fill=blue!10, minimum width=2.5cm, align=center]
            \tikzstyle{salle}=[circle, draw, minimum width=1.5cm, align=center]

            \node[entree,label=45:0] (E0) {Entrée};
            \node[salle,label=45:1 , right=of E0] (M) {M};
            \node[salle,label=45:2 , right=of M] (B) {B};
            \node[salle,label=45:3 , right=of B] (J) {J};
            \node[salle,label=45:5 , below=of M] (D) {D \\ Trésor};
            \node[salle,label=45:4 , below=of J] (U) {U};

            \draw (E0) to node [near start,sloped,above] {\fld} (M);
            \draw (M) to node [near start,sloped,above] {\fld} (D);
            \draw (M) to node [near start,sloped,above] {\fld} (B);
            \draw (B) to node [near start,sloped,above] {\fld} (J);
            \draw (J) to node [near start,sloped,above] {\fld} (U);
            \draw [bend right=45](B) to (U);
        \end{tikzpicture}

    \newpage
    % Donjon 2
    \subsection{Donjon 2}
        %Rose des vents
        \begin{tikzpicture}[rotate=90, scale=0.35, every node/.style={scale=0.6}]
            \foreach   \direction/\label in {0/N, 90/E, 180/S, 270/W}{
                \node[fill=green!10] at ( -\direction : 15mm) (\label) {\label};
            }
            \draw[arrows={->[scale=1]}] (N) -- (S);
            \draw[arrows={->[scale=1]}] (S) -- (N);
            \draw[arrows={->[scale=1]}] (E) -- (W);
            \draw[arrows={->[scale=1]}] (W) -- (E);
        \end{tikzpicture}

        \bigskip
        {\bf Parcours en largeur}

        \begin{tikzpicture}
            \tikzstyle{entree}=[circle, thick, draw=blue!75, fill=blue!10, minimum width=2.5cm, align=center]
            \tikzstyle{salle}=[circle, draw, minimum width=1.5cm, align=center]

            \node[entree, label=135:0 ,label=45:0] (E0) {Entrée};
            \node[salle, label=135:1 ,label=45:2, right=of E0] (C) {C};
            \node[salle, label=135:1 ,label=45:1, above=of C] (J) {J};
            \node[salle, label=135:1 ,label=45:3, below right=of C] (X) {X};
            \node[salle, label=135:2 ,label=45:5, above right=of X] (S) {S \\ Trésor};
            \node[salle, label=135:2 ,label=45:4, above=of S] (B) {B};
            \node[salle, label=135:3 ,label=45:6, right=of S] (H) {H};
            \node[salle, label=135:2 ,label=45:7, below=of X] (G) {G \\ Trésor};

            \draw (E0.north) .. controls +(up:1.25cm) .. node [very near start,sloped,above] {\fld} (J.west);
            \draw (E0) to node [near start,sloped,above] {\fld} (C);
            \draw (E0.south) .. controls +(down:0.75cm) .. node [very near start,sloped,below] {\fld} (X.west);
            \draw (J) to node [near start,sloped,above] {\fld} (B);
            \draw (C) to node [near start,sloped,above] {\fld} (S);
            \draw (S) to (H);
            \draw (B.east) .. controls +(right:1.75cm) .. (H.north);
            \draw (X) to node [near start,sloped,above] {\fld} (G);
            \draw (H.south) .. controls +(down:1.25cm) .. node [pos=0.95 ,sloped,below] {\fld} (X.east);
        \end{tikzpicture}

        \bigskip
        {\bf Parcours en profondeur}\\
        Nombre de portes traversées: \\
        En suivant l'algorithme : 12 \\
        Au mieux : 10
        \bigskip

        \begin{tikzpicture}
            \tikzstyle{entree}=[circle, thick, draw=blue!75, fill=blue!10, minimum width=2.5cm, align=center]
            \tikzstyle{salle}=[circle, draw, minimum width=1.5cm, align=center]

            \node[entree,label=45:0] (E0) {Entrée};
            \node[salle,label=45:7, right=of E0] (C) {C};
            \node[salle,label=45:1, above=of C] (J) {J};
            \node[salle,label=45:4, below right=of C] (X) {X};
            \node[salle,label=45:6, above right=of X] (S) {S \\ Trésor};
            \node[salle,label=45:2, above=of S] (B) {B};
            \node[salle,label=45:3, right=of S] (H) {H};
            \node[salle,label=45:5, below=of X] (G) {G \\ Trésor};

            \draw (E0.north) .. controls +(up:1.25cm) .. node [very near start,sloped,above] {\fld} (J.west);
            \draw (J) to node [near start,sloped,above] {\fld} (B);
            \draw (E0) to (C);
            \draw (C) to node [near end,sloped,above] {\flg} (S);
            \draw (S) to node [near end,sloped,above] {\flg} (H);
            \draw (B.east) .. controls +(right:1.75cm) .. node [near start,sloped,above] {\fld} (H.north);
            \draw (X) to node [near start,sloped,above] {\fld} (G);
            \draw (E0.south) .. controls +(down:0.75cm) .. (X.west);
            \draw (H.south) .. controls +(down:1.25cm) .. node [very near start ,sloped,below] {\flg} (X.east);
        \end{tikzpicture}

    \newpage
    % Donjon 3
    \subsection{Donjon 3}
        %Rose des vents
        \begin{tikzpicture}[rotate=90, scale=0.35, every node/.style={scale=0.6}]
            \foreach   \direction/\label in {0/N, 90/E, 180/S, 270/W}{
                \node[fill=green!10] at ( -\direction : 15mm) (\label) {\label};
            }
            \draw[arrows={->[scale=1]}] (N) -- (S);
            \draw[arrows={->[scale=1]}] (S) -- (N);
            \draw[arrows={->[scale=1]}] (E) -- (W);
            \draw[arrows={->[scale=1]}] (W) -- (E);
        \end{tikzpicture}

        \bigskip
        {\bf Parcours en largeur}

        \begin{tikzpicture}
            \tikzstyle{entree}=[circle, thick, draw=blue!75, fill=blue!10, minimum width=2.5cm, align=center]
            \tikzstyle{salle}=[circle, draw, minimum width=1.5cm, align=center]

            \node[entree, label=135:0 ,label=45:0] (E0) {Entrée};
            \node[salle, label=135:1 ,label=45:1, right=of E0] (A) {A};
            \node[salle, label=135:2 ,label=45:2, above=of A] (T) {T};
            \node[salle, label=135:2 ,label=45:3, right=of A] (F) {F};
            \node[salle, label=135:3 ,label=45:5, left=of T] (H) {H \\ Trésor};
            \node[salle, label=135:3 ,label=45:4, right=of T] (C) {C};
            \node[salle, label=135:3 ,label=45:7, right=of F] (G) {G \\ Trésor};
            \node[salle, label=135:3 ,label=45:6, right=of C] (N) {N};

            \draw (H) to node [near end,sloped,above] {\flg} (T);
            \draw (T) to node [near start,sloped,above] {\fld} (C);
            \draw (A) to node [near start,sloped,above] {\fld} (T);
            \draw (C) to (N);
            \draw (E0) to node [near start,sloped,above] {\fld} (A);
            \draw (A) to node [near start,sloped,above] {\fld} (F);
            \draw (F) to node [near start,sloped,above] {\fld} (G);
            \draw (N.south) .. controls +(down:0.75cm) and +(up:0.75cm) .. node [pos=0.95,sloped,above] {\fld} (F.north);
        \end{tikzpicture}

        \bigskip
        {\bf Parcours en profondeur}\\
        Nombre de portes traversées: \\
        En suivant l'algorithme : 13 \\
        Au mieux : 10
        \bigskip

        \begin{tikzpicture}
            \tikzstyle{entree}=[circle, thick, draw=blue!75, fill=blue!10, minimum width=2.5cm, align=center]
            \tikzstyle{salle}=[circle, draw, minimum width=1.5cm, align=center]
            \node[entree,label=45:0] (E0) {Entrée};
            \node[salle,label=45:1, right=of E0] (A) {A};
            \node[salle,label=45:2, above=of A] (T) {T};
            \node[salle,label=45:5, right=of A] (F) {F};
            \node[salle,label=45:7, left=of T] (H) {H \\ Trésor};
            \node[salle,label=45:3, right=of T] (C) {C};
            \node[salle,label=45:6, right=of F] (G) {G \\ Trésor};
            \node[salle,label=45:4, right=of C] (N) {N};

            \draw (H) to node [near end,sloped,above] {\flg} (T);
            \draw (T) to node [near start,sloped,above] {\fld} (C);
            \draw (A) to node [near start,sloped,above] {\fld} (T);
            \draw (C) to node [near start,sloped,above] {\fld} (N);
            \draw (E0) to node [near start,sloped,above] {\fld} (A);
            \draw (A) to (F);
            \draw (F) to node [near start,sloped,above] {\fld} (G);
            \draw (N.south) .. controls +(down:0.75cm) and +(up:0.75cm) .. node [near start,sloped,above] {\flg} (F.north);
        \end{tikzpicture}

    \newpage
    % Donjon 4
    \subsection{Donjon 4}
        %Rose des vents
        \begin{tikzpicture}[rotate=90, scale=0.35, every node/.style={scale=0.6}]
            \foreach   \direction/\label in {0/N, 90/E, 180/S, 270/W}{
                \node[fill=green!10] at ( -\direction : 15mm) (\label) {\label};
            }
            \draw[arrows={->[scale=1]}] (N) -- (S);
            \draw[arrows={->[scale=1]}] (S) -- (N);
            \draw[arrows={->[scale=1]}] (E) -- (W);
            \draw[arrows={->[scale=1]}] (W) -- (E);
        \end{tikzpicture}

        \bigskip
        {\bf Parcours en largeur}

        \begin{tikzpicture}
            \tikzstyle{entree}=[circle, thick, draw=blue!75, fill=blue!10, minimum width=2.5cm, align=center]
            \tikzstyle{salle}=[circle, draw, minimum width=1.5cm, align=center]
            \node[entree, label=135:0 ,label=45:0] (E0) {Entrée};
            \node[salle, label=135:1 ,label=45:1, right=of E0] (N) {N};
            \node[salle, label=135:2 ,label=45:4, right=of N] (B) {B \\ Trésor};
            \node[salle, label=135:2 ,label=45:3, above=of B] (Y) {Y};
            \node[salle, label=135:1 ,label=45:2, below=of N] (D) {D};
            \node[salle, label=135:2 ,label=45:5, right=of D] (E) {E};
            \node[salle, label=135:2 ,label=45:6, below=of D] (J) {J};

            \draw (E0) to node [near start,sloped,above] {\fld} (N);
            \draw (N) to node [near start,sloped,above] {\fld} (B);
            \draw (N) [bend left=45] to node [very near start,sloped,above] {\fld} (Y);
            \draw (B) to (Y);
            \draw (B.south) .. controls +(down:0.75cm) and +(up:0.75cm) .. (D.north);
            \draw (E0) [bend right=40] to node [very near start,sloped,below] {\fld} (D);
            \draw (D) to node [near start,sloped,above] {\fld} (E);
            \draw (D) to node [near start,sloped,below] {\fld} (J);
            \draw (J) [bend right=45] to (E);
        \end{tikzpicture}

        \bigskip
        {\bf Parcours en profondeur}\\
        Nombre de portes traversées: \\
        En suivant l'algorithme : 13 \\
        Au mieux : 11
        \bigskip

        \begin{tikzpicture}
            \tikzstyle{entree}=[circle, thick, draw=blue!75, fill=blue!10, minimum width=2.5cm, align=center]
            \tikzstyle{salle}=[circle, draw, minimum width=1.5cm, align=center]
            \node[entree,label=45:0] (E0) {Entrée};
            \node[salle,label=45:1, right=of E0] (N) {N};
            \node[salle,label=45:3, right=of N] (B) {B \\ Trésor};
            \node[salle,label=45:2, above=of B] (Y) {Y};
            \node[salle,label=45:4, below=of N] (D) {D};
            \node[salle,label=45:5, right=of D] (E) {E};
            \node[salle,label=45:6, below=of D] (J) {J};

            \draw (E0) to node [near start,sloped,above] {\fld} (N);
            \draw (N) to (B);
            \draw (N) [bend left=45] to node [very near start,sloped,above] {\fld} (Y);
            \draw (B) to node [near end,sloped,below] {\flg} (Y);
            \draw (B.south) .. controls +(down:0.75cm) and +(up:0.75cm) .. node [very near start,sloped,below] {\flg} (D.north);
            \draw (E0) [bend right=40] to (D);
            \draw (D) to node [near start,sloped,above] {\fld} (E);
            \draw (D) to (J);
            \draw (E) [bend left=45] to node [very near start,sloped,below] {\flg} (J);
        \end{tikzpicture}

    \newpage
    % Donjon 5
    \subsection{Donjon 5}
        %Rose des vents
        \begin{tikzpicture}[rotate=90, scale=0.35, every node/.style={scale=0.6}]
            \foreach   \direction/\label in {0/N, 90/E, 180/S, 270/W}{
                \node[fill=green!10] at ( -\direction : 15mm) (\label) {\label};
            }
            \draw[arrows={->[scale=1]}] (N) -- (S);
            \draw[arrows={->[scale=1]}] (S) -- (N);
            \draw[arrows={->[scale=1]}] (E) -- (W);
            \draw[arrows={->[scale=1]}] (W) -- (E);
        \end{tikzpicture}

        \bigskip
        {\bf Parcours en largeur}

        \begin{tikzpicture}
            \tikzstyle{entree}=[circle, thick, draw=blue!75, fill=blue!10, minimum width=2.5cm, align=center]
            \tikzstyle{salle}=[circle, draw, minimum width=1.5cm, align=center]
            \node[entree, label=135:0, label=45:0] (E0) {Entrée};
            \node[salle, label=135:1, label=45:1, right=of E0] (G) {G};
            \node[salle, label=135:2, label=45:3, above=of G] (U) {U};
            \node[salle, label=135:3, label=45:6, left= 1.5cm of U] (A) {A};
            \node[salle, label=135:3, label=45:5, right=of U] (J) {J \\ Trésor};
            \node[salle, label=135:2, label=45:4, below=of G] (F) {F};
            \node[salle, label=135:1, label=45:2, left= 1.5cm of F] (R) {R};

            \draw (E0) to node [near start, sloped, above] {\fld} (G);
            \draw (G) to node [near start, sloped, above] {\fld} (U);
            \draw (G) to node [near start, sloped, above] {\fld} (F);
            \draw (U) to node [near end, sloped, above] {\fld} (A);
            \draw (U) to node [near start, sloped, above] {\fld} (J);
            \draw (E0) to node [sloped, above] {\fld} (R);
            \draw (R) to (F);
            \draw (J.south) .. controls +(down:4.25cm) .. (F.east);
        \end{tikzpicture}

        \bigskip
        {\bf Parcours en profondeur}\\
        Nombre de portes traversées: \\
        En suivant l'algorithme : 13 \\
        Au mieux : 10
        \bigskip

        \begin{tikzpicture}
            \tikzstyle{entree}=[circle, thick, draw=blue!75, fill=blue!10, minimum width=2.5cm, align=center]
            \tikzstyle{salle}=[circle, draw, minimum width=1.5cm, align=center]
            \node[entree, label=45:0] (E0) {Entrée};
            \node[salle, label=45:1, right=of E0] (G) {G};
            \node[salle, label=45:2, above=of G] (U) {U};
            \node[salle, label=45:6, left= 1.5cm of U] (A) {A};
            \node[salle, label=45:3, right=of U] (J) {J \\ Trésor};
            \node[salle, label=45:4, below=of G] (F) {F};
            \node[salle, label=45:5, left= 1.5cm of F] (R) {R};

            \draw (E0) to node [near start, sloped, above] {\fld} (G);
            \draw (G) to node [near start, sloped, above] {\fld} (U);
            \draw (G) to (F);
            \draw (U) to node [near end, sloped, above] {\fld} (A);
            \draw (U) to node [near start, sloped, above] {\fld} (J);
            \draw (E0) to (R);
            \draw (R) to node [near end, sloped, above] {\flg} (F);
            \draw (J.south) .. controls +(down:4.25cm) .. node [pos=0.025, sloped, below] {\flg} (F.east);
        \end{tikzpicture}

\end{document}
